\section{Behavior, Interaction and Priority (BIP)}
\label{sec:bip}
We recall the necessary concepts of the BIP framework~\cite{bip2}.
BIP allows to construct systems by superposing three layers of design: Behavior, Interaction, and Priority.
The \emph{behavior} layer consists of a set of atomic components represented by transition systems. 
The \emph{interaction} layer provides the collaboration between components. 
Interactions are described using sets of ports. 
The \emph{priority} layer is used to specify scheduling policies applied to the interaction layer, given by a strict partial order on interactions.

\subsection{Atomic Components}
We define \emph{atomic components} as transition systems with a set of ports labeling individual transitions. These ports are used for communication between different components.

\begin{definition}[Atomic Component]
An  {\em atomic component} $B$ is a labeled transition system represented by a triple $(Q,P,\goesto)$ where $Q$ is a set of {\em states}, $P$ is a set of {\em communication ports}, $\goesto\, \subseteq Q\times P\times Q$ is a set of {\em possible transitions}, each labeled by some port.
\end{definition}

For any pair of states $q,q'\in Q$ and a port $p\in P$, we write $q \goesto[p] q'$, iff $(q,p,q')\in\,\goesto$. When the communication port is irrelevant, we simply write $q \goesto q'$. Similarly, $q \goesto[p]$ means that there exists $q'\in Q$ such that $q \goesto[p] q'$. In this case, we say that $p$ is \emph{enabled} in state $q$.

In practice, atomic components are extended with variables. Each variable may be
bound to a port and modified through interactions involving this port. We also
associate a guard and an update function (i.e., action) to each transition. A guard is a
predicate on variables that must be true to allow the execution of the
transition. An update function is a local computation triggered by the
transition that modifies the variables. 

Figure \ref{fig:philo-fork} shows an atomic component $P$ that corresponds to the behavior of a philosopher in the dining-philosopher problem, where $Q = \{e,h\}$ denotes eating and hungry, $P = \{\release, \get\}$ denotes releasing and getting of forks, and $\goesto = \{e \goesto[\release] h, h \goesto[\get] e\}$.


\subsection{Composition Component}
For a given system built from a set of $n$ atomic components $\{B_i = (Q_i, P_i, \goesto_i)\}_{i=1}^n$, we assume that their respective sets of ports are pairwise disjoint, i.e., for any two $i\not= j$ from $\set{1..n}$, we have $P_i \cap P_j = \emptyset$. We can therefore define the set $P = \bigcup_{i=1}^n P_i$ of all ports in the system. An {\em interaction} is a set $a \subseteq P$ of ports. When we write $a = \{p_i\}_{i\in I}$, we suppose that for $i \in I$, $p_i \in P_i$, where $I \subseteq \set{1..n}$.

Similar to atomic components, BIP extends interactions by associating a guard
and a transfer function to each of them. Both the guard and the function are defined over
the variables that are bound to the ports of the interaction. The guard must be
true to allow the interaction. When the interaction takes place, the
associated transfer function is called and modifies the variables.

\begin{figure}[t]
  \begin{center}
    \mbox{
      \subfigure[Philosopher $P$ and fork $F$ atomic components.]{\label{fig:philo-fork}\scalebox{0.35}{\input{figs/philAndFork.pdf_t}}} \quad
      \subfigure[Dining philosophers composite component with four philosophers.]{\label{fig:diningbip}\scalebox{0.3}{\input{figs/diningbip.pdf_t}}}
      }
    \caption{Dining philosophers.}
    \label{fig:diningSpectrum}
  \end{center}
\end{figure}

\begin{definition}[Composite Component]\label{def.bip.composition}
A {\em composite \linebreak component} (or simply {\em component}) is defined by a composition operator parameterized by a set of interactions $\Gamma \subseteq 2^P$.  $B \stackrel{\mathit{def}}{=} \Gamma(B_1,\dots,B_n)$, is a transition system $(Q,\Gamma, \goesto)$, where $Q=\bigotimes_{i=1}^n Q_i$ and $\goesto$ is the least set of transitions satisfying the rule

\begin{mathpar}
\inferrule
{
    a = \{p_i\}_{i\in I}\in \Gamma\\
    \forall i\in I: q_i \goesto[p_i]_i q'_i\\
    \forall i\not\in I: q_i = q'_i\\
}
{
    \bm{q} = (q_1,\dots,q_n) \goesto[a] \bm{q'} = (q'_1,\dots,q'_n)
}
\end{mathpar}
\end{definition}

The inference rule says that a composite component $B=\Gamma(B_1,\dots,B_n)$ can
execute an interaction $a\in \Gamma$, iff for each port $p_i\in a$, the
corresponding atomic component $B_i$ can execute a transition labeled with
$p_i$; the states of components that do not participate in the interaction stay
unchanged. 

Figure~\ref{fig:diningbip} illustrates a composite component $\Gamma(P_0, P_1, P_2, P_3, F_0, F_1, F_2, F_3)$, where each $P_i$ (resp. $F_i$)  is identical to component $P$ (resp. $F$) in Figure
\ref{fig:philo-fork} and $\Gamma = \Gamma_\get \, \cup \, \Gamma_\release$, where $\Gamma_\get = \bigcup_{i=0}^{3} \{P_i.\get, F_i.\use_l, F_{(i+1) \% 4}.use_r\}$ and $\Gamma_\release = \bigcup_{i=0}^{3}\{P_i.\release, F_i.\free_l, F_{(i+1)\%4}.free_r\}$. 


Notice that several distinct interactions can be enabled at the same time, thus introducing non-determinism in the product behavior. One can add priorities to reduce non-determinism. In this case, one of the interactions with the highest priority is chosen non-deterministically.

\begin{definition}[Priority]
  \label{defn:priority}
  Let $C = (Q,\Gamma,\goesto)$ be the behavior of the composite component $\Gamma(\{B_1, \ldots, B_n\})$.  A {\em priority model} $\prio$ is a
  strict partial order on the set of interactions $\Gamma$. Given a priority model $\prio$, we
  abbreviate $(a,a')\in \prio$ by $a \prec a'$. Adding the priority model $\prio$ over $\Gamma(\{B_1, \ldots, B_n\})$ defines a new composite component $B = \prio\big(\Gamma(\{B_1, \ldots, B_n\})\big)$ noted $\pi(C)$ and whose behavior is defined by $(Q, \Gamma, \goesto_\prio)$, where $\goesto_\prio$ is the least set of transitions satisfying the following rule:
\begin{mathpar}
\inferrule*
	{
      \bm{q} \goesto[a] \bm{q'} \and
      \neg\big(\exists a'\in \Gamma,\exists \bm{q''}\in Q: a \prec a' \wedge \bm{q} \goesto[a'] \bm{q''} \big)
    }
    {
      \bm{q} \goesto[a]_\prio \bm{q'}
    }
\end{mathpar}
\end{definition}
%
An interaction $a$ is enabled in $\pi(C)$ whenever $a$ is enabled in $C$ and $a$ is maximal according to $\pi$ among the active interactions in $C$.


BIP provides both centralized and distributed implementations. In the centralized implementation,
a centralized engine guarantees to execute only one interaction at a time, and thus conforms to the operational semantics of the BIP.  The main loop of the BIP engine consists of the following steps:
(1) Each atomic component sends to the engine its current
location; (2) The engine enumerates the list of interactions in the system,
selects the enabled ones based on the current location
of the atomic components and eliminates the ones
with low priority; (3) The engine non-deterministically selects an interaction
out of the enabled interactions; (4) Finally, the engine notifies the corresponding components
and schedules their transitions for execution.

Alternatively, BIP allows the generation of distributed
implementations~\cite{bip-distributed} where non-conflicting interactions
can be simultaneously executed.

\begin{definition}[BIP system]
\label{def:bipsystem}
A BIP system is a tuple $(B,  \bm{q_0})$, where $ \bm{q_0}$ is the initial state with $ \bm{q_0} \in \bigotimes_{i=1}^n Q_i$ being the tuple of initial states of atomic components.
\end{definition}


For the rest of the paper, we fix an arbitrary BIP-system $(B,  \bm{q_0})$, where  $B = \prio\big(\Gamma(\{B_1, \ldots, B_n\})\big)$ with semantics $C = (Q, \Gamma, \goesto)$.

We abstract the execution of a BIP system as a trace.
%
\begin{definition}[BIP trace]
\label{def:trace-global}
A BIP trace $\rho = ( \bm{q_0} \cdot a_0 \cdot  \bm{q_1} \cdot a_1 \cdots  \bm{q_{i-1}} \cdot a_{i-1} \cdot  \bm{q_i})$ is an alternating sequence of states of $Q$ and interactions in $\Gamma$; where $q_k \xrightarrow{a_k}  \bm{q_{k+1}} \in \rightarrow$, for $k \in [0, i-1]$.
\end{definition}

Given a trace $\rho = ( \bm{q_0} \cdot a_0 \cdot  \bm{q_1} \cdot a_1 \cdots  \bm{q_{i-1}} \cdot a_{i-1} \cdot  \bm{q_i})$, 
$\rho^{ \bm{q_i}}$ (resp. $\rho^a_i$) denotes the $i^\text{th}$ state (resp. interaction) of the trace, i.e., $ \bm{q_i}$ (resp. $a_i$). Also, $\bm{\rho}(C)$ denotes the set of all the traces of an LTS $C$. 
